\documentclass[11pt]{article}

\usepackage{amsmath}
\usepackage{dsfont}
\usepackage{booktabs}
\usepackage{tablefootnote}
\usepackage{parskip}
\usepackage[bitstream-charter]{mathdesign}
\usepackage{microtype}

\renewcommand{\baselinestretch}{1.2}

\title{Basic SIR Modeling Idea}
\author{
  T.~Shafer\\
  Elder Research, Inc.\\
  \texttt{tom.shafer@elderresearch.com}
}

\begin{document}
\maketitle

\section{SIR Equations}

\begin{equation}
\begin{aligned}
    \dot S(t) &= -\beta \frac{I(t)}{N} S(t) \\
    \dot I(t) &=  \beta \frac{I(t)}{N} S(t) - \gamma I(t) \\
    \dot R(t) &=  \gamma I(t)
\end{aligned}
\end{equation}

In these equations, $\beta$ is the strength of the interactions between the \textit{susceptible} subset of the population $S(t)$ and the fraction of the \textit{infected} proportion of the population $I(t)$, and $\gamma$ is the rate of removal (either by death, recovery, perfect quarantine, etc.) of infected individuals. As Wayne noted, these are \textit{average} strengths assuming perfect mixing between populations, which isn't remotely true.

\section{SIR Modeling}

Consider turning the differential equation for $I(t)$ into a daily difference equation:
$$
\begin{aligned}
I(T+1) &= I(T) + \beta \frac{I(T)}{N} S(T) - \gamma I(T) \\
 &= I(T) \left[ 1 + \beta S(T) / N - \gamma \right].
\end{aligned}
$$

We ought to be able to predict the number of new infected persons on day $T+1$ as proportional to the number on day $T$. The proportionality factor involves the SIR parameters $\beta$ and $\gamma$ as well as the susceptible population density $S(T)/N$.

We don't have $\beta$ or $\gamma$ in the model, but we can fit them as long as we have $S(T)$, which we do: our data \textit{appears to track} $\mathcal{I}(T) = \sum_t^T I(t) $, the cumulative count of infections. Following the SIR theme, then, $S(t) \equiv N - \mathcal{I}(t)$ and we have all the parameters we need to fit a model.

\section{Research Program}

\begin{enumerate}
  \item Build a simple, average model with \textit{lme4} or \textit{brms}. Maybe only includes records after infections are nonzero.
  \item Experiment with models at the monthly or weekly level, rolling over time, to capture time-varying estimates of $\beta$ and $\gamma$.
\end{enumerate}

If we can capture $\beta(t)$ and $\gamma(t)$, then we can (1) compute $R_0 = \beta/\gamma$ and (2) estimate whether the policy changes affected these disease vectors. It's not likely to work, but if it did it would be very clean.

\end{document}
